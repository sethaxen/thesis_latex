\documentclass[../main.tex]{subfiles}

\begin{document}

\begin{refsection}

\todo{introduce section, e.g. pointing to further reading}

\section{Theoretical underpinnings of kinematic ensemble representations}\label{theoretical-underpinnings-of-kinematic-ensemble-representations}

In order to construct a kinematic ensemble representation, we must first introduce kinematics and the kinematic groups that appear therein.
Next we must introduce the concept of probability distributions on those groups.
Finally, we will end with harmonic analysis on kinematic groups.
Subsequently, we will construct the kinematic ensemble representation and demonstrate how specific measurements can be simulated from such a representation.

\subsection{Introduction to kinematics}\label{introduction-to-kinematics}

\emph{Kinematics} deals with the motion of geometric objects \cite{chirikjian_harmonic_2016}.
We will limit ourselves to a description of \emph{rigid body kinematics}, which regards shape-preserving motions of points and bodies.
Central to rigid body kinematics, which we will henceforth refer to only as \emph{kinematics}, are the \emph{kinematic groups}: \emph{translations}, \emph{rotations}, and \emph{rigid motions}.
Because humans have an intuitive understanding of these groups, it is useful to recall their properties before introducing them formally.

A \emph{translation} shifts a point in some direction by some distance; in other words, translations are vectors that act upon a particle at a point to move it to a new point.
A subsequent translation composes with the first by addition; that is, translating by first one vector and then another is equivalent to translating at once by the sum of the two vectors.
Any translation can be inverted simply by translating in the opposite direction.
A translation by no distance at all doesn't move the particle.

Analogously, a \emph{rotation} acts on a point or an orientation to produce a new point or orientation, respectively.
A \emph{rigid motion} acts on rigid frames to move them rigidly through space.
Every rigid motion can be described with a rotation followed by a translation.

\subsection{Manifolds}\label{manifolds}

The canonical example of a manifold is a sphere.
Though we often think of the sphere as a three-dimensional object, it's actually a two-dimensional object embedded in a three-dimensional space.
This becomes apparent once you constrain a point to the surface of the sphere.
On earth, we navigate using the two directions of north and south.
Because we are quite small compared to the earth, we like the ancients perceive that we dwell on a flat plane.
But even as we take a global view, we can come up with many ways to treat the planet as flat, as cartographers have been doing for centuries.

These informal concepts generalize well to other sets formed by nonlinear constraints.
A manifold is a topological set that can be locally flattened.
That is, we can cover the set entirely with overlapping \emph{open sets} (neighborhoods) in which we define a map from the manifold to a Euclidean (flat) space.
This property is especially important when considering \emph{smooth manifolds}, where these maps are differentiable, such that we can smoothly transition from one neighborhoods set of coordinates to that of another overlapping neighborhood \cite{lee_introduction_2003}.
This enables us to use the familiar machinery of calculus on these complicated spaces.

\subsection{Tangent spaces}\label{tangent-spaces}

Consider again the planet earth, and consider many curves on the surface of earth passing through a given point.
At exactly that point, for each curve, draw its velocity vector.
These velocities do not exist on the surface of the earth.
In fact, the collection of all such velocity vectors form a plane perfectly tangent to the earth at exactly that point.
This is a general property of manifolds, that at every point on the manifold there exists a tangent space in which all tangent vectors (such as velocities) exist.
The tangent spaces at nearby points have no relationship to one another without endowing the manifold with more structure.
Unintuitively, there is also a cotangent space, where cotangent vectors (such as momentum) exist.
As we will not rely on cotangent spaces in this work, we will forego a more detailed explanation of cotangent spaces.

\subsection{Groups}\label{groups}

A group $\group{G} = (\mset{S}, \circ)$ is a set $\mset{S}$ together with a binary operation $\circ$ that has several special properties.
Formally, the properties are:
\begin{enumerate}
\def\labelenumi{\arabic{enumi}.}
\item
  Composability: $(f \circ g) \in \group{G}$
\item
  Associativity: $(f \circ g) \circ h = f \circ (g \circ h)$, for all $f,g,h \in \group{G}$
\item
  Identity: $f \circ e = e \circ f = f$, where $e \in \group{G}$ is the identity element
\item
  Invertibility: $f \circ f^{-1} = f^{-1} \circ f = e$, where $f^{-1} \in \group{G}$ is the inverse of $f$
\end{enumerate}
Informally, elements of the group can act on each other to produce new elements of the group, these actions can be undone, and there's such a thing as a null action (the identity).
That all group elements can act on each other produces a kind of symmetry in the manifold of the group.
However, there is always one privileged point, the identity element.
Properties of the identity element and its tangent space can often be generalized to all elements of the group.
All groups we will consider here are called \emph{Lie groups}, groups whose underlying set is also a smooth manifold, where the group operation and inverse are also differentiable.

\subsection{Lie algebras, group exponential, and group logarithm}\label{lie-algebras-group-exponential-and-group-logarithm}

The tangent space at the identity element of a Lie group $\group{G}$ is called the \emph{Lie algebra} $\liealg{g}$ of the group.
Every element $X$ of the Lie algebra can be mapped to an element $g$ of the group using the \emph{group exponential} map: $\exp \colon \liealg{g} \to \group{G}.$
The exponential map is the unique map that satisfies the identity
\[X = \lim_{t \to 0} \frac{d}{dt} \exp(t X)\]
for all $X \in \liealg{g}$ and $t \in \bbR$.
In some neighborhood around the origin of the Lie algebra, the exponential map can be inverted by the \emph{group logarithm} map:
\[\log \colon \group{G} \to \liealg{g}.\]
These maps should not be confused with either the scalar exponential and logarithm functions or with the Riemannian exponential or logarithm maps.

\todo{discuss group actions?}

\subsection{Relevant examples of Lie groups}\label{relevant-examples-of-lie-groups}

\subsubsection{The translation group}\label{the-translation-group}

The \emph{translation group} is defined as $\Transg{n} = (\bbR^n, +)$.
That is, it is $n$-dimensional real space together with the operation of addition.
Translations are vectors.
It is left to the reader to verify that translation vectors have the four properties of a group.
The translation group is commutative.
For all $x,y \in \Transg{n}, x \circ y = y \circ x$.
That is, order of translation does not matter.
Also, its group exponential and logarithm maps are just addition and subtraction, respectively.

\subsubsection{General linear group}\label{general-linear-group}

All of the kinematic groups we will consider can be formulated as a Lie group whose set is the set of all matrices under the binary operation of matrix multiplication.
The properties of these groups immediately follow from this definition.
If any matrix in the group can be multiplied by all others in either direction, then they all must be square $n \times n$ matrices.
They must also be invertible, and the identity element is always the identity matrix.
This is called the \emph{general linear group} $\GL{n}$ of $n \times n$ matrices.

It can be shown for all subgroups of $\GL{n}$ that the group exponential map is the \emph{matrix exponential}, defined by its power series for a matrix $\mat{X}$ as
\[
\expm \mat{X} = \sum_{n=0}^\infty \frac{1}{n!} X^n
              = I + X + \frac{1}{2} X^2 + \frac{1}{6} X^3 + \ldots.
\]

Likewise, the group logarithm map for $\GL{n}$ is the \emph{matrix logarithm}, defined analogously as
\[
\logm \mat{X} = \sum_{n=1}^\infty (-1)^{n-1} \frac{1}{n} (X - I)^n
              = X - I - \frac{1}{2} (X - I)^2 + \frac{1}{3} (X - I)^3 - \ldots.
\]
Both the matrix exponential and matrix logarithm have as a special case the familiar scalar exponential and logarithm for $1 \times 1$ matrices.

Just like we can obtain the set of all invertible real numbers (positive non-zero numbers) by exponentiating all real numbers, the Lie algebra $\gl{n}$ of $\GL{n}$ is the set of all real $n \times n$ matrices.

\subsubsection{General affine group}\label{general-affine-group}

The \emph{general affine group} $\GA{n}$ is a subgroup of $\GL{n+1}$ consisting of \emph{affine transformations}, which are transformations that may stretch, reflect, rotate, or move an object but preserve properties like the straightness of lines and parallel lines.
When acting on real space $\bbR^n$, $\GA{n}$ is the \emph{semidirect product} of the translation group and the $\GL{n}$.
That is, we can represent its group action as the action of an element of $\GL{n}$ followed by the action of an element of $\Transg{n}$:
\[(\vec{v}, \mat{X}) \circ (\vec{w}, \mat{Y}) = (\mat{X} \vec{w} + \vec{v}, \mat{X} \mat{Y}),\]
where $\mat{X},\mat{Y} \in \GL{n}$, and $\vec{v}, \vec{w} \in \Transg{n}$.

$\GA{n}$ can be written as $n + 1 \times n + 1$ matrices whose group operation is matrix multiplication:
\[
\begin{pmatrix}
\mat{X} & \vec{v}\\
\trans{\vec{0}} & 1
\end{pmatrix}
\begin{pmatrix}
\mat{Y} & \vec{w}\\
\trans{\vec{0}} & 1
\end{pmatrix} = 
\begin{pmatrix}
\mat{X} \mat{Y} & \mat{X} \vec{w} + \vec{v}\\
\trans{\vec{0}} & 1
\end{pmatrix}.
\]

Because the translation group is a subgroup of the general affine group, we can also construct the translation group as a group under matrix multiplication, where only the right-most column is non-zero.

\subsubsection{The special orthogonal group}\label{the-special-orthogonal-group}

The group of rotation matrices in $n$-dimensional space is called the \emph{special orthogonal group}, written $\SO{n}$.
It is formally defined as
\[\SO{n}=\{\mat{R} \in \bbR^{n \times n} \mid \trans{\mat{R}} \mat{R} = \mat{R} \trans{\mat{R}} = \mat{I}_n, \det\mat{R} = +1 \},\]
where $\trans{\cdot}$ is the matrix transpose, $\mat{I}_n$ us the $n \times n$ identity matrix (with zeros everywhere except the diagonal elements, which are all 1), and $\det(\cdot)$ is the matrix determinant.
The group operation of $\SO{n}$ is matrix multiplication, the inverse operation is the matrix transpose, and the identity element is the identity matrix.

We can obtain the Lie algebra $\so{n}$ by differentiating either of the constraints $\mat{R}(t) \trans{\mat{R}(t)} = \trans{\mat{R}(t)} \mat{R}(t) = \mat{I}$, for all $t \in \bbR$.
\[\frac{d}{dt} (\mat{R}(t) \trans{\mat{R}(t)}) = \left(\frac{d}{dt}\mat{R}(t)\right) \trans{\mat{R}(t)} + \mat{R}(t) \trans{\left(\frac{d}{dt}\mat{R}(t) \right)} = \mat{0}\]
\[\left(\frac{d}{dt}\mat{R}(t)\right) \trans{\mat{R}(t)} = -\trans{\left(\left(\frac{d}{dt}\mat{R}(t)\right) \trans{\mat{R}(t)}\right)}\]

Taking the limit as $t$ goes to 0, we get

\[\lim_{t \to 0} \left(\frac{d}{dt}\mat{R}(t)\right) \trans{\mat{R}(t)} = \mat{X} \mat{I} = - \trans{\mat{X} \mat{I}}, \quad \mat{X} = -\trans{\mat{X}}\]

That is, $\so{n}$ consists of all skew-symmetric matrices, matrices equal to the negative of their own transpose.
For the case of $\so{3}$, the Lie algebra of the $3d$ rotation matrices, these skew symmetric matrices have only three independent elements, which when collected into a vector are called the angular velocity.
The product of one of these matrices with a vector of length 3 is equivalent to the cross-product.
The matrix exponential and matrix logarithm for $\SO{3}$ can be computed efficiently and are well-known \cite{chirikjian_harmonic_2016}.

\subsubsection{The special Euclidean group}\label{the-special-euclidean-group}

The group of rigid transformations in $n$-dimensional space is called the \emph{special Euclidean group}, which is written $\SE{n}$.
It is a subgroup of the general affine group, where it is the semidirect product of $\SO{n}$ and $\Transg{n}$.

\subsection{Analogs to the normal distribution on manifolds}\label{analogs-to-the-normal-distribution-on-manifolds}

When selecting a generic distribution to represent a kinematic motion, it is attractive to look for one that has some of the intuitive and useful properties of the multivariate normal distribution in Euclidean space ($\bbR^n$).
Some of these properties are
\begin{enumerate}
\def\labelenumi{\arabic{enumi}.}
\item
  It maximizes the entropy (i.e.~is the most uniform distribution) subject only to the constraints of a specific mean and covariance.
\item
  It is entirely specified with a mean that is also its mode (point of highest density) and a covariance.
\item
  It is symmetric about its mean.
\item
  It is the solution to the \emph{heat equation}. That is, it is the distribution followed by the endpoint of a Brownian motion trajectory with position-independent drift evaluated for some fixed time.
\end{enumerate}
Unfortunately, there is in general no analog to the normal distribution on a manifold that satisfies all of the above properties.
However, there are a few analogs to the normal distribution that we might consider:
\begin{enumerate}
\def\labelenumi{\arabic{enumi}.}
\item
  A projected-normal distribution on an embedded manifold is a distribution in the ambient space of the manifold that is then orthogonally projected onto the manifold.
\item
  A restricted-normal distribution on an embedded manifold consists of a normal distribution in the ambient space intersected with the manifold.
\item
  A tangent-normal distribution is a normal distribution in the tangent space of the manifold at a \emph{mean} point pushed forward to the manifold with the necessary change of variables.
\item
  An intrinsic normal distribution is a normal distribution expressed in exponential coordinates and is the maximum entropy distribution subject to the existence of the Riemannian center of mass and a notion of concentration. \cite{pennec_intrinsic_2006}
\item
  A diffusive normal distribution is the solution to the heat equation on the manifold \todo{expand, define heat equation or completely remove its mention}.
  That is, if a Dirac measure (point mass) is permuted to diffuse across the manifold without drift or force, the resulting distribution at a fixed stopping time is the diffusive normal.
\end{enumerate}
Each of the above distributions may be useful under different circumstances.
However, most of them do not have a known closed form of the normalized density function and therefore do not permit computation of expectations.
The remainder only permit computation of expectations using quadrature or expensive Markov Chain Monte Carlo (MCMC) techniques; neither of these are feasible in the inner loop of another sampling scheme.
For kinematic groups, the exception is the diffusive normal distribution, which can be represented as weighting functions of the harmonic basis functions of the kinematic groups.
To represent these weighting functions, we will first introduce harmonic analysis with a focus on harmonic bases of the kinematic groups.

\subsection{Harmonic analysis on kinematic groups}\label{harmonic-analysis-on-kinematic-groups}

The field of harmonic analysis is concerned with representing functions as the weighted superposition of \emph{harmonic basis functions} (or \emph{harmonics}), which can be thought of informally as waves of different frequencies \cite{chirikjian_harmonic_2016}.
The most well known basis functions are the Fourier bases that appear in the Fourier series on the circle and the Fourier transform on the real line.
We first consider these harmonics and their extension to the kinematic groups at a high level.

\todo{need to settle on one set of verbs ("in harmonics", "in basis", "projected onto basis", etc.}

\subsubsection{Fourier series}\label{fourier-series}

\todo{EXPLAIN WHY WE'RE COVERING FOURIER SERIES IN DETAIL HERE}

The circle group $\torus = \{z \in \bbC, |z| = 1\}$ is the group consisting of all points on the circle of radius 1 on the complex plane with the group operation of complex multiplication.
That is, each point also corresponds to an angle of in-plane rotation from the $x$-axis, which we can explicitly write as $z(\theta) = e^{\im\theta} = \cos\theta + \im \sin\theta$, for any $\theta \in \bbR$.
In fact the real line is the Lie algebra of the circle group.
Multiplying one point by another effectively rotates the points:
\[e^{\im\theta_1} e^{\im\theta_2} = e^{\im (\theta_1 + \theta_2)}\]
Hence, there is a diffeomorphic (smooth and 1-to-1) map from the circle group to the 2D rotation group $SO(2)$, which rotates points in the $xy$-plane. \todo{EXPLAIN}

The complex Fourier series is a Laurent series (power series) on $\torus$.
Consider a function $f \in L^2(\torus)$ \todo{EXPLAIN BELOW} on the circle.
This function can be written as an infinite series in terms of the complex Fourier basis functions $U_n(z)$:
\[
\begin{aligned}
  U_n(z) &\doteq z^n \\
  \coef{f}_n &= \inner{f}{U_n}_\torus
              = \int_\torus f(z) \hconj{U_n(z)} d\mu(z)\\
  f(z) &= \sum_{n=-\infty}^\infty \coef{f}_n U_n(z),\\
\end{aligned}
\]
where $\inner{\cdot}{\cdot}_\manifold{M}$ is the \emph{inner product} of the functions on the manifold $\manifold{M}$, $\conj{\cdot}$ is the complex conjugate (i.e.~$\conj{(a + \im b)} = a - \im b$), and the integral is with respect to the Haar (uniform) measure $\mu(z)$ on the circle (i.e.~$\int_\torus d\mu(z) = 1$).
The basis functions may be more familiar when written as trigonometric functions with angle $x \in \bbR$:
\[U_n(z) = z^n = (e^{\im x})^n = \cos (nx) + \im \sin(nx).\]
The harmonics are examples of \emph{representations} of the group.
That is, they have the same properties as elements of the group $z,w,y \in \torus$:
\[
  U_n(z)U_n(w) = U_n(zw)\\
  U_n(\conj{z}) = U_n(z^{-1}) = U_n(z)^{-1} = \hconj{U_n(z)}\\
  U_n(z) \hconj{U_n(z)} = \hconj{U_n(z)} U_n(z) = 1 = U_n(1)\\
  U_n(z) \bigl(U_n(w) U_n(y)\bigr) = \bigl(U_n(z) U_n(w)\bigr) U_n(y)
\]
In this case, they are not only representations, they are actually elements of the group.

They are also \emph{orthogonal} to each other:
\[\inner{U_m}{U_n}_\torus = \int_\torus U_n(z) \hconj{U_m(z)} d\mu(z) = \delta_{mn}.\]
where $\delta_{mn} = \begin{cases} &1 \text{ if } m = n\\ &0 \text{ if } m \ne n\end{cases}$ is the Kronecker delta function.
That is, their inner product is 0 unless they have the same index.

This gives rise to the convolution theorem.
Given two functions $f,g \in L^2(\torus)$, their \emph{convolution} $h = f \convwith g \in L^2(\torus)$ can be written in terms of the coefficients:
\begingroup
\allowdisplaybreaks
\begin{align*}
h(w) &= (f \convwith g)(w) \doteq \int_\torus f(z) g(\conj{z} w) d\mu(z)\\
&= \int_\torus
    \left(
      \sum_{n=-\infty}^\infty \coef{f}_n U_n(z)
    \right)
    \left(
      \sum_{m=-\infty}^\infty \coef{g}_m U_m(\conj{z} w)
    \right)
    d\mu(z)\\
&= \int_\torus
   \sum_{n=-\infty}^\infty \sum_{m=-\infty}^\infty
    \coef{f}_n U_n(z)
    \coef{g}_m U_m(\conj{z} w)
    d\mu(z)\\
&= \sum_{n=-\infty}^\infty \sum_{m=-\infty}^\infty
    \coef{f}_n \coef{g}_m U_m(w)
    \int_\torus
    U_n(z)
    \hconj{U_m(z)}
    d\mu(z)\\
&= \sum_{n=-\infty}^\infty \sum_{m=-\infty}^\infty
    \coef{f}_n \coef{g}_m U_m(w)
    \delta_{mn}\\
&= \sum_{n=-\infty}^\infty \coef{f}_n \coef{g}_n U_n(w)\\
h_n &= \inner{f \convwith g}{U_n}_\torus = \coef{f}_n \coef{g}_n
\end{align*}
\endgroup
That is, the projection of $h$ onto the harmonic basis can be obtained by straightforward multiplication of the coefficients of $f$ and $g$.
An intuitive understanding of convolution arises when $f$ and $g$ are probability density functions on the circle:
\[
\conj{z} w \sim g\\
z \sim f\\
z \conj{z} w = w \sim h = f \convwith h
\]
That is, $h$ is the distribution that arises by perturbing random draws from $g$ with random draws from $f$.

The Fourier coefficients and harmonics are functions of the discrete index $n$, but they can also be written as infinite-dimensional diagonal matrices, a notation that will be convenient when considering the groups $\SO{3}$ and $SE(3)$ below:
\[
U(z) = \diag \begin{pmatrix}\ldots, U_{-1}(z), U_0(z), U_1(z), \ldots\end{pmatrix}\\
\coef{F} = \inner{f}{U}_\torus = \diag\begin{pmatrix}\ldots, \coef{f}_{-1}, \coef{f}_0, \coef{f}_1, \ldots\end{pmatrix}\\
f(z) = \tr (\coef{F} U(z))
\]
For computation, the infinite-dimensional diagonal matrix is truncated at some finite index $N$, which corresponds to setting all higher order terms to 0.

Although the other harmonics we will consider are more complicated than that of the circle group, several important properties generalize (they are orthogonal, they are group representations, and convolution is coefficient multiplication).

\subsubsection{Fourier transform}\label{fourier-transform}

The harmonics for the 1-dimensional translation group $\transg(1) \equiv \bbR$ are the Fourier basis functions used in the Fourier transform:
\[
\begin{aligned}
  U(x; \xi) &= e^{i \xi x}\\
  f(x) &= \int_\bbR \coef{f}(\xi) U(x;\xi)\\
  \coef{f}(\xi) &= \frac{1}{2\pi} \int_\bbR f(x) \hconj{U(x; \xi)} d\mu(x),\\
\end{aligned}
\]
where $d\mu(x)=dx$ is the uniform measure on $\bbR$, and $\xi$ is a pseudo-frequency term.

The harmonics of the Fourier transform have a similar orthogonality property as the harmonics of the Fourier series.
However, a key distinction between the transform and the series is that the transform is a continuous function of a pseudo-frequency term $\xi$, while the series is a discrete function of an order term.
Briefly, this difference arises from the compact, periodic nature of the circle vs the non-compact nature of real space.

\subsubsection{Characteristic functions}\label{characteristic-functions}

Harmonics, such as the basis functions in the Fourier series and Fourier transform, are examples of harmonic bases, or harmonics.
Harmonics are useful because they enable the approximation of potentially complicated functions.
For example, the low order terms (low $|n|$) in the Fourier series correspond to low frequency, high period trigonometric functions; therefore, the corresponding Fourier coefficients capture global or low-resolution information about the function.
As the number of terms used increases, the resulting approximate function captures increasingly more fine-grained features of the function.
Probability distributions may be approximately represented by their Fourier coefficients or their Fourier transform \todo{UNIFY}.
The resulting function is called the characteristic function.
Characteristic functions have the convenient property that the first term is always 1, which arises from the distribution having a total probability of 1.
One such characteristic function will be considered later.

\todo{EXPAND WITH EXAMPLE USING FOURIER SERIES, PERHAPS WRAPPED NORMAL DISTRIBUTION, WHICH INTRODUCES DIFFUSIVE NORMAL COVERED LATER}

\subsubsection{\texorpdfstring{Harmonics on $\SO{3}$}{Harmonics on \textbackslash SO\{3\}}}\label{harmonics-on-so3}

The harmonics of $\SO{3}$ are conceptually similar to the above harmonics on the circle, though they are functions of three discrete indices instead of one.
They are widely studied and used in quantum mechanics, where they are known as the integer order Wigner-$D$ matrices.
See \cite{varshalovich_wigner_1988} for a description of Wigner-$D$ matrices, and (Choi et al. \cite{choi_rapid_1999,boyle_angular_2013} for algorithms for constructing the matrices.

The $\SO{3}$ harmonics are \footnotemark \todo{explain what they are}

\footnotetext{There are several different conventions for Wigner $D$-matrices, mostly differing in the proportionality constant used and the Condon-Shortley phase factor $(-1)^{m-n}$. We use identical conventions as \cite{varshalovich_wigner_1988} and ensure that all formulas in this work are internally consistent.}

\[
\coef{f}^\ell_{mn} \doteq \inner{f}{U^\ell_{mn}}_{\SO{3}} =
  \int_{\SO{3}} f(R) \hconj{U^\ell_{mn}(R)} d\mu(R)\\
f(R) = \sum_{\ell=0}^\infty (2\ell+1) \sum_{m=-\ell}^{\ell} \sum_{n=-\ell}^{\ell} f^\ell_{mn} U^\ell_{mn}(R)\\
\inner{ U^\ell_{mn} }{ U^s_{pq} }_{\SO{3}} = \frac{1}{2\ell+1} \delta_{\ell s} \delta_{mp} \delta_{nq}
\]
where $d\mu(R)$ is the uniform measure on $\SO{3}$.
The $\SO{3}$ harmonics can be framed as an infinite dimensional block-diagonal matrix with blocks of order $\ell$ of increasing size $(2\ell+1, 2\ell+1)$:
\[
\coef{F} \doteq \inner{f}{U}_{\SO{3}} =
\begin{pmatrix}
  \coef{F_1} &            0 &             0 &  \dots \\
           0 &   \coef{F_2} &             0 &  \dots \\
           0 &            0 &    \coef{F_3} &  \dots \\
      \vdots &       \vdots &        \vdots & \ddots \\
\end{pmatrix}\\
\coef{F}_0 = \begin{pmatrix}f^0_{00}\end{pmatrix}\\
\coef{F}_1 = \begin{pmatrix}
f^\ell_{-1,-1} & f^\ell_{-1,0} & f^\ell_{-1,1}\\
 f^\ell_{0,-1} &  f^\ell_{0,0} &  f^\ell_{0,1}\\
 f^\ell_{1,-1} &  f^\ell_{1,0} &  f^\ell_{1,1}
\end{pmatrix}\\
f(R) = \tr(\coef{F} \hconj{U}),
\]
where $\hconj{\cdot} = \trans{\conj{\cdot}}$ is the Hermitian or conjugate transpose, and $\tr(\cdot)$ is the trace function (the sum of the matrix diagonal).

\subsubsection{\texorpdfstring{Harmonics on $\SE{3}$}{Harmonics on \textbackslash SE\{3\}}}\label{harmonics-on-se3}

Harmonics on $\SE{3}$ are less well studied than those of $\SO{3}$ or $\transg{3}$.
However, they combine features of both sets of harmonics.
Given a transformation $T = (t,R) = (t, I_3) \circ (0, R) \in SE(3)$, the harmonics and coefficients are written
\[
U(T; \xi) = U(t, I; \xi) U(0, R; \xi)
\]
The exception for the kinematic groups is the diffusive normal distribution, where the density on $\SE{3}$ can be expressed using the irreducible unitary representations (IURs) of the group, the generalization of Fourier coefficients to $\SE{3}$ (SI,REF).
Therefore,

\subsection{\texorpdfstring{The diffusive normal distribution on $\SE{3}$}{The diffusive normal distribution on \textbackslash SE\{3\}}}\label{the-diffusive-normal-distribution-on-se3}

\section{The kinematic ensemble representation (KER)}\label{the-kinematic-ensemble-representation-ker}

\section{Simulating data from kinematic ensembles}\label{simulating-data-from-kinematic-ensembles}

\subsection{NOEs}\label{noes}

\subsection{Order parameters}\label{order-parameters}

\section{Ensemble inference}

\printbibliography[heading=subbibintoc]
\end{refsection}

\end{document}